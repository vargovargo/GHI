\def\handout{0}   % set to 1 to produce 4-up handouts instead of slides
\def\notes{0}     % set to 1 to show \note{}s
\ifnum\handout=1  % see above for an alternative which uses two preamble files
\documentclass[handout,13pt,compress,c]{beamer}
\usepackage{pgfpages}
\pgfpagesuselayout{4 on 1}[letterpaper,landscape,border shrink=4mm]
\setbeamertemplate{footline}[page number]   % omit if don't want slide number at bottom right
\else
\documentclass[13pt,compress,c]{beamer}
\fi
\usetheme{PaloAlto}
\usepackage{graphicx}
\usepackage{natbib}          % for author year citations \citet \citep
\usepackage{relsize}         % for \smaller etc.
\usepackage{xcolor}
\usepackage{listings}
\lstset{language=bash,
        basicstyle=\tiny,
        frame=single,
        backgroundcolor=\color{lightgray},
        commentstyle=\color{black},
        showstringspaces=false
        }
\DeclareGraphicsExtensions{.pdf, .jpg, .png}
\setbeamercolor{normal text}{bg=blue!5}
\setbeamertemplate{footline}[page number] % omit if don't want slide number at bottom right
\ifnum\notes=1 \setbeameroption{show notes} \fi
\newcommand{\ft}[1]{\frametitle{#1}}
\newcommand{\bi}{\begin{itemize}}
\newcommand{\ei}{\end{itemize}}
\newcommand{\figw}[2]{\centerline{\includegraphics[width=#2\textwidth]{#1}}}
\newcommand{\figh}[2]{\centerline{\includegraphics[height=#2\textheight]{#1}}}
\newcommand{\fig}[1]{\centerline{\includegraphics{#1}}}
\newcommand{\foot}[1]{\footnotetext{#1}} % smaller text in bottom margin, e.g. citations
\makeatletter
\renewcommand\@makefntext[1]{\noindent#1} % see p. 114 of LaTeX Companion 2nd edition
\makeatother
\renewcommand\footnoterule{}
\def\newblock{\hskip .11em plus .33em minus .07em}
\title[ITHIM]{A Discussion about ITHIM and Its Future Implementation}
\author[Younkin et al.]{\small{Samuel Younkin, Jason Vargo, Maggie Grabow, Jonathan Patz}}
\institute{Global Health Institute\\University of Wisconsin-Madison}
\date{\tiny{June 2016}}
%~~~~~~~~~~~~~~~~~~~~~~~~~~~~~~~~~~~~~~~~~~~~~~~~~~~~~~~~~~~~~~~~~~~~~~~~~~
%~~~~~~~~~~~~~~~~~~~~~~~~~~~~~~~~~~~~~~~~~~~~~~~~~~~~~~~~~~~~~~~~~~~~~~~~~~
%~~~~~~~~~~~~~~~~~~~~~~~~~~~~~~~~~~~~~~~~~~~~~~~~~~~~~~~~~~~~~~~~~~~~~~~~~~
\begin{document}
\frame{\titlepage}
%~~~~~~~~~~~~~~~~~~~~~~~~~~~~~~~~~~~~~~~~~~~~~~~~~~~~~~~~~~~~~~~~~~~~~~~~~~
%~~~~~~~~~~~~~~~~~~~~~~~~~~~~~~~~~~~~~~~~~~~~~~~~~~~~~~~~~~~~~~~~~~~~~~~~~~
%~~~~~~~~~~~~~~~~~~~~~~~~~~~~~~~~~~~~~~~~~~~~~~~~~~~~~~~~~~~~~~~~~~~~~~~~~~
\frame{
   \ft{Outline}
\tableofcontents
}
%~~~~~~~~~~~~~~~~~~~~~~~~~~~~~~~~~~~~~~~~~~~~~~~~~~~~~~~~~~~~~~~~~~~~~~~~~~
%~~~~~~~~~~~~~~~~~~~~~~~~~~~~~~~~~~~~~~~~~~~~~~~~~~~~~~~~~~~~~~~~~~~~~~~~~~
%~~~~~~~~~~~~~~~~~~~~~~~~~~~~~~~~~~~~~~~~~~~~~~~~~~~~~~~~~~~~~~~~~~~~~~~~~~
\section{Introduction}
%~~~~~~~~~~~~~~~~~~~~~~~~~~~~~~~~~~~~~~~~~~~~~~~~~~~~~~~~~~~~~~~~~~~~~~~~~~
%~~~~~~~~~~~~~~~~~~~~~~~~~~~~~~~~~~~~~~~~~~~~~~~~~~~~~~~~~~~~~~~~~~~~~~~~~~
%~~~~~~~~~~~~~~~~~~~~~~~~~~~~~~~~~~~~~~~~~~~~~~~~~~~~~~~~~~~~~~~~~~~~~~~~~~
\begin{frame}{Introduction}
Samuel G. Younkin, Ph.D.
\bi\item R programmer from the World of Biostats \& Epi.
\item Contracting with GHI for Summer
\item C.V.
\bi \item Research Associate: UW-Madison, Department of Biostatistics
\& Medical Informatics
\item Postdoctoral Fellow: Johns Hopkins Bloomberg, School of Public Health
\item Ph.D.: Case Western Reserve University, School of Medicine
\item \url{https://github.com/syounkin}
\ei\ei
\end{frame}
%~~~~~~~~~~~~~~~~~~~~~~~~~~~~~~~~~~~~~~~~~~~~~~~~~~~~~~~~~~~~~~~~~~~~~~~~~~
%~~~~~~~~~~~~~~~~~~~~~~~~~~~~~~~~~~~~~~~~~~~~~~~~~~~~~~~~~~~~~~~~~~~~~~~~~~
%~~~~~~~~~~~~~~~~~~~~~~~~~~~~~~~~~~~~~~~~~~~~~~~~~~~~~~~~~~~~~~~~~~~~~~~~~~
\section{ITHIM}
%~~~~~~~~~~~~~~~~~~~~~~~~~~~~~~~~~~~~~~~~~~~~~~~~~~~~~~~~~~~~~~~~~~~~~~~~~~
%~~~~~~~~~~~~~~~~~~~~~~~~~~~~~~~~~~~~~~~~~~~~~~~~~~~~~~~~~~~~~~~~~~~~~~~~~~
%~~~~~~~~~~~~~~~~~~~~~~~~~~~~~~~~~~~~~~~~~~~~~~~~~~~~~~~~~~~~~~~~~~~~~~~~~~
\begin{frame}[fragile]
\frametitle{ITHIM}
\bi \item The ITHIM Model is composed of three exposures
\bi \item Active Transport, $\mathrm{AT}$, $\frac{\mathrm{MET}\cdot\mathrm{hrs}}{\mathrm{week}}$
\item Air Particulates, $\mathrm{PM}_{25}$, units?
\item Road Exposure, $x_\mathrm{road}$, $\frac{\mathrm{person}\cdot\mathrm{miles}}{\mathrm{week}}$
\ei
\item ITHIM models a collection of responses (including road injuries)
  and sums over disease to provide an overall burden of
  disease
\ei
\end{frame}
%~~~~~~~~~~~~~~~~~~~~~~~~~~~~~~~~~~~~~~~~~~~~~~~~~~~~~~~~~~~~~~~~~~~~~~~~~~
%~~~~~~~~~~~~~~~~~~~~~~~~~~~~~~~~~~~~~~~~~~~~~~~~~~~~~~~~~~~~~~~~~~~~~~~~~~
%~~~~~~~~~~~~~~~~~~~~~~~~~~~~~~~~~~~~~~~~~~~~~~~~~~~~~~~~~~~~~~~~~~~~~~~~~~
\subsection{Active Transport}
%~~~~~~~~~~~~~~~~~~~~~~~~~~~~~~~~~~~~~~~~~~~~~~~~~~~~~~~~~~~~~~~~~~~~~~~~~~
%~~~~~~~~~~~~~~~~~~~~~~~~~~~~~~~~~~~~~~~~~~~~~~~~~~~~~~~~~~~~~~~~~~~~~~~~~~
%~~~~~~~~~~~~~~~~~~~~~~~~~~~~~~~~~~~~~~~~~~~~~~~~~~~~~~~~~~~~~~~~~~~~~~~~~~
\begin{frame}[fragile]
\frametitle{Active Transport}
\bi
\item Breast Cancer, Colon Cancer, Depression, Dementia
\bi\item How do we find good estimates for $\varphi_1$ and $k$?
\bi\item See slide \ref{RR}\ei
\item Do these estimates vary by region?
\item Should we use disease-specific estimates for $k$?
\ei 
\ei
\end{frame}
%~~~~~~~~~~~~~~~~~~~~~~~~~~~~~~~~~~~~~~~~~~~~~~~~~~~~~~~~~~~~~~~~~~~~~~~~~~
%~~~~~~~~~~~~~~~~~~~~~~~~~~~~~~~~~~~~~~~~~~~~~~~~~~~~~~~~~~~~~~~~~~~~~~~~~~
%~~~~~~~~~~~~~~~~~~~~~~~~~~~~~~~~~~~~~~~~~~~~~~~~~~~~~~~~~~~~~~~~~~~~~~~~~~
\begin{frame}[fragile]
\frametitle{Active Transport}
\bi
\item Diabetes, IHD, Hypertensive Heart Disease, Lung Cancer, Stroke
\bi\item Why is non-Travel METs not included in the exposure variable
for these diseases?\ei
\ei
\end{frame}
%~~~~~~~~~~~~~~~~~~~~~~~~~~~~~~~~~~~~~~~~~~~~~~~~~~~~~~~~~~~~~~~~~~~~~~~~~~
%~~~~~~~~~~~~~~~~~~~~~~~~~~~~~~~~~~~~~~~~~~~~~~~~~~~~~~~~~~~~~~~~~~~~~~~~~~
%~~~~~~~~~~~~~~~~~~~~~~~~~~~~~~~~~~~~~~~~~~~~~~~~~~~~~~~~~~~~~~~~~~~~~~~~~~
\begin{frame}[fragile]
\frametitle{Active Transport}
\bi\item Why is cycling MET expenditure fixed at 6 across age
group and sex while walking MET expenditure is a function of walking
speed? 
\bi\item See Slide \ref{METs}\ei
\ei
\end{frame}
%~~~~~~~~~~~~~~~~~~~~~~~~~~~~~~~~~~~~~~~~~~~~~~~~~~~~~~~~~~~~~~~~~~~~~~~~~~
%~~~~~~~~~~~~~~~~~~~~~~~~~~~~~~~~~~~~~~~~~~~~~~~~~~~~~~~~~~~~~~~~~~~~~~~~~~
%~~~~~~~~~~~~~~~~~~~~~~~~~~~~~~~~~~~~~~~~~~~~~~~~~~~~~~~~~~~~~~~~~~~~~~~~~~
\subsection{Air Pollution}
\begin{frame}[fragile]
\frametitle{Air Pollution}
\bi\item How do we estimate exposure (baseline and scenario)?
\item How do we model risk to exposure relationship?
\bi\item Does the exponential parameter $k$ vary by region or
disease?\ei
\item Is the $\mathrm{PM}_{25}$ exposure a function of $x_{\mathrm{road}}$?
\ei
\end{frame}
%~~~~~~~~~~~~~~~~~~~~~~~~~~~~~~~~~~~~~~~~~~~~~~~~~~~~~~~~~~~~~~~~~~~~~~~~~~
%~~~~~~~~~~~~~~~~~~~~~~~~~~~~~~~~~~~~~~~~~~~~~~~~~~~~~~~~~~~~~~~~~~~~~~~~~~
%~~~~~~~~~~~~~~~~~~~~~~~~~~~~~~~~~~~~~~~~~~~~~~~~~~~~~~~~~~~~~~~~~~~~~~~~~~
\subsection{Road Injuries}
%~~~~~~~~~~~~~~~~~~~~~~~~~~~~~~~~~~~~~~~~~~~~~~~~~~~~~~~~~~~~~~~~~~~~~~~~~~
%~~~~~~~~~~~~~~~~~~~~~~~~~~~~~~~~~~~~~~~~~~~~~~~~~~~~~~~~~~~~~~~~~~~~~~~~~~
%~~~~~~~~~~~~~~~~~~~~~~~~~~~~~~~~~~~~~~~~~~~~~~~~~~~~~~~~~~~~~~~~~~~~~~~~~~
\begin{frame}[fragile]
\frametitle{Road Injuries}
\bi\item Best source for regional data on road accidents? 
\item How do we estimate mutipliers?\ei
\end{frame}
%~~~~~~~~~~~~~~~~~~~~~~~~~~~~~~~~~~~~~~~~~~~~~~~~~~~~~~~~~~~~~~~~~~~~~~~~~~
%~~~~~~~~~~~~~~~~~~~~~~~~~~~~~~~~~~~~~~~~~~~~~~~~~~~~~~~~~~~~~~~~~~~~~~~~~~
%~~~~~~~~~~~~~~~~~~~~~~~~~~~~~~~~~~~~~~~~~~~~~~~~~~~~~~~~~~~~~~~~~~~~~~~~~~
\section{R Implementation}
%~~~~~~~~~~~~~~~~~~~~~~~~~~~~~~~~~~~~~~~~~~~~~~~~~~~~~~~~~~~~~~~~~~~~~~~~~~
%~~~~~~~~~~~~~~~~~~~~~~~~~~~~~~~~~~~~~~~~~~~~~~~~~~~~~~~~~~~~~~~~~~~~~~~~~~
%~~~~~~~~~~~~~~~~~~~~~~~~~~~~~~~~~~~~~~~~~~~~~~~~~~~~~~~~~~~~~~~~~~~~~~~~~~
\begin{frame}[fragile]
\frametitle{R Package, \textit{ITHIM}}
\bi
\item \url{https://github.com/syounkin/ITHIM}
\item Create a list that defines the scenario/baseline model
\bi \item Define parameters \item Compute means, stratified by age and sex \item Retrieve quantiles of active transport time (lognormal with constant $c_v$)\ei
\ei
\begin{semiverbatim}
\begin{lstlisting}
> ITHIM <- list(
  parameters = parameters <- createParameterList(),
  means = means <- computeMeanMatrices(parameters),
  quintiles = quintiles <- getQuintiles(means)
)
\end{lstlisting}
\end{semiverbatim}
\end{frame}
%~~~~~~~~~~~~~~~~~~~~~~~~~~~~~~~~~~~~~~~~~~~~~~~~~~~~~~~~~~~~~~~~~~~~~~~~~~
%~~~~~~~~~~~~~~~~~~~~~~~~~~~~~~~~~~~~~~~~~~~~~~~~~~~~~~~~~~~~~~~~~~~~~~~~~~
%~~~~~~~~~~~~~~~~~~~~~~~~~~~~~~~~~~~~~~~~~~~~~~~~~~~~~~~~~~~~~~~~~~~~~~~~~~
\begin{frame}[fragile]
\frametitle{ITHIM Objects}
\begin{semiverbatim}
\begin{lstlisting}
> ITHIM.baseline <- list(
  parameters = parameters <- createParameterList(baseline = TRUE),
  means = means <- computeMeanMatrices(parameters),
  quintiles = quintiles <- getQuintiles(means)
)

> ITHIM.scenario <- list(
  parameters = parameters <- createParameterList(baseline = FALSE),
  means = means <- computeMeanMatrices(parameters),
  quintiles = quintiles <- getQuintiles(means)
)
\end{lstlisting}
\end{semiverbatim}
\end{frame}
%~~~~~~~~~~~~~~~~~~~~~~~~~~~~~~~~~~~~~~~~~~~~~~~~~~~~~~~~~~~~~~~~~~~~~~~~~~
%~~~~~~~~~~~~~~~~~~~~~~~~~~~~~~~~~~~~~~~~~~~~~~~~~~~~~~~~~~~~~~~~~~~~~~~~~~
%~~~~~~~~~~~~~~~~~~~~~~~~~~~~~~~~~~~~~~~~~~~~~~~~~~~~~~~~~~~~~~~~~~~~~~~~~~
\begin{frame}[fragile]
\frametitle{AF Breast Cancer}
\begin{semiverbatim}
\begin{lstlisting}
> comparativeRisk <- compareModels(ITHIM.baseline,ITHIM.scenario)
\end{lstlisting}
\end{semiverbatim}
\begin{semiverbatim}
\begin{lstlisting}
> with(comparativeRisk, {
  as.data.frame(lapply(AF[c("BreastCancer","Depression")],
    function(x) 100*round(x[-(1:2),],4))
  )
)

##           BreastCancer.M BreastCancer.F Depression.M Depression.F
## ageClass3              0           1.92         2.36         3.19
## ageClass4              0           1.61         2.97         2.67
## ageClass5              0           1.77         2.96         2.93
## ageClass6              0           2.65         2.53         4.36
## ageClass7              0           1.61         1.95         2.67
## ageClass8              0           0.95         1.26         1.57
\end{lstlisting}
\end{semiverbatim}
\end{frame}
%~~~~~~~~~~~~~~~~~~~~~~~~~~~~~~~~~~~~~~~~~~~~~~~~~~~~~~~~~~~~~~~~~~~~~~~~~~
%~~~~~~~~~~~~~~~~~~~~~~~~~~~~~~~~~~~~~~~~~~~~~~~~~~~~~~~~~~~~~~~~~~~~~~~~~~
%~~~~~~~~~~~~~~~~~~~~~~~~~~~~~~~~~~~~~~~~~~~~~~~~~~~~~~~~~~~~~~~~~~~~~~~~~~
\section{Appendix}
\subsection{Relative Risk vs.\ Exposure}
\begin{frame}[fragile]
\label{RR}
\frametitle{Relative Risk vs.\ Exposure}
We model the relative risk, $\varphi(x^\prime)$, as a function of
transformed exposure.
\begin{equation}
\varphi\left(\tilde{x}\right) = e^{r \tilde{x}}
\end{equation}
\bi
\item $r=\log\left(\varphi_1\right)$ \bi\item$\varphi_1$ is ``RR 1 MET'' in the EXCEL workbook\ei
\item $\tilde{x}=x^k$, $k=0.25,0.375,0.5,\ldots$
\ei
\bi
\item How can we update the estimates for $\varphi_1$ and $k$?
\item Is it better to vary $k$ by disease?\ei
\end{frame}
%~~~~~~~~~~~~~~~~~~~~~~~~~~~~~~~~~~~~~~~~~~~~~~~~~~~~~~~~~~~~~~~~~~~~~~~~~~
%~~~~~~~~~~~~~~~~~~~~~~~~~~~~~~~~~~~~~~~~~~~~~~~~~~~~~~~~~~~~~~~~~~~~~~~~~~
%~~~~~~~~~~~~~~~~~~~~~~~~~~~~~~~~~~~~~~~~~~~~~~~~~~~~~~~~~~~~~~~~~~~~~~~~~~
\subsection{METs}
\begin{frame}[fragile]
\label{METs}
\frametitle{METs}
\bi
\item $\mathbf{\mathrm{MET}}_\mathrm{cycling} = 6\ \mathrm{METs}$
\item $\mathbf{\mathrm{MET}}_\mathrm{walking} = f\left( \mu_s  \right) \mathrm{METs}$
\ei
\begin{equation}
f(\mu_s) = \begin{cases} 1.2\mu_s + 0.08 &\mbox{if } \mu_s \ge 2.02\ \mathrm{kph} \\
2.5 & \mbox{if } \mu_s < 2.02\ \mathrm{kph}\end{cases}
\end{equation}
\bi
\item Why are the two treated differently?
\item Should we model cycling METs as a function of cycling speed?
\item Note that model exposure is in terms of $\frac{\mathrm{MET}\cdot\mathrm{hrs}}{\mathrm{week}}$
\ei
\end{frame}
%~~~~~~~~~~~~~~~~~~~~~~~~~~~~~~~~~~~~~~~~~~~~~~~~~~~~~~~~~~~~~~~~~~~~~~~~~~
%~~~~~~~~~~~~~~~~~~~~~~~~~~~~~~~~~~~~~~~~~~~~~~~~~~~~~~~~~~~~~~~~~~~~~~~~~~
%~~~~~~~~~~~~~~~~~~~~~~~~~~~~~~~~~~~~~~~~~~~~~~~~~~~~~~~~~~~~~~~~~~~~~~~~~~
%% \subsection{$\mathrm{RR}_\mathrm{1MET}$}
%% \begin{frame}[fragile]
%% \frametitle{$\mathrm{RR}_\mathrm{1MET}$}
%% \bi
%% \item Physical Act. RRs worksheet contains values for $\varphi_1$, given fixed $k$ for each disease
%% \item $\varphi_1$ is estimated using values for $x_{\textrm{lit}}$ and $\varphi(x_{\textrm{lit}})$ and a fixed $k$
%% \ei
%% \end{frame}
\end{document}
