% Nulla mi mi, venenatis sed ipsum varius, Table~\ref{table1} volutpat euismod diam. Proin rutrum vel massa non gravida. Quisque tempor sem et dignissim rutrum. Lorem ipsum dolor sit amet, consectetur adipiscing elit. Morbi at justo vitae nulla elementum commodo eu id massa. In vitae diam ac augue semper tincidunt eu ut eros. Fusce fringilla erat porttitor lectus cursus, vel sagittis arcu lobortis. Aliquam in enim semper, aliquam massa id, cursus neque. Praesent faucibus semper libero.

% % Place tables after the first paragraph in which they are cited.
% \begin{table}[!ht]
% \begin{adjustwidth}{-2.25in}{0in} % Comment out/remove adjustwidth environment if table fits in text column.
% \centering
% \caption{
% {\bf Table caption Nulla mi mi, venenatis sed ipsum varius, volutpat euismod diam.}}
% \begin{tabular}{|l+l|l|l|l|l|l|l|}
% \hline
% \multicolumn{4}{|l|}{\bf Heading1} & \multicolumn{4}{|l|}{\bf Heading2}\\ \thickhline
% $cell1 row1$ & cell2 row 1 & cell3 row 1 & cell4 row 1 & cell5 row 1 & cell6 row 1 & cell7 row 1 & cell8 row 1\\ \hline
% $cell1 row2$ & cell2 row 2 & cell3 row 2 & cell4 row 2 & cell5 row 2 & cell6 row 2 & cell7 row 2 & cell8 row 2\\ \hline
% $cell1 row3$ & cell2 row 3 & cell3 row 3 & cell4 row 3 & cell5 row 3 & cell6 row 3 & cell7 row 3 & cell8 row 3\\ \hline
% \end{tabular}
% \begin{flushleft} Table notes Phasellus venenatis, tortor nec vestibulum mattis, massa tortor interdum felis, nec pellentesque metus tortor nec nisl. Ut ornare mauris tellus, vel dapibus arcu suscipit sed.
% \end{flushleft}
% \label{table1}
% \end{adjustwidth}
% \end{table}

We used the National Household Transportation Survey to estimate the
mean walk and cycle times for individuals that live in metropolitan
areas for each of seven states (CA, TX, NY, FL, VA, NC and AZ)
\cite{NHTS}.  These states were selected becaused they had sufficient
sample size for reasonable estimates of age-sex specific
travel-related walking and cycling means, Figure \ref{meanMatrices}.

Of these seven states one was found to show active transport behavior
that, if adopted nation-wide, would avert 75-100,000 DALYs overall.
Values for national disease-specific DALYs were found using the CDC
Wonder database \cite{CDCWonder}.  California stood out among the
estimates for national DALYs averted.  Most of the decrease in DALYs
is due to prevention of dementia in the oldest age group, $80+$ years
old.  California exhibits much greater walking and cycling means in
this age group than the national average. In particular we see the
cycling mean We also see prevention of depression and diabetes in the
middle age groups due to increased active transport time.
