Regular exercise has been shown to provide health benefits in terms of
chronic disease prevention \cite{warburton2006}, cognitive function
\cite{hillman2008}, and overall well being and satisfaction with life
\cite{maher2013}. Still only half of Americans obtain the recommended
levels of aerobic activity, which is 2.5 hours per week. There is
considerable variability in the attainment of this goal by location
within the country. Supportive local built and policy environments are
important determinants of population-level activity
patterns. Particularly, as they facilitate and foster routine and moderate
physical activity, such as walking, local decisions on design and
policy can play a huge role in improving the nation's health.

Given the constraints and complexity of local governance and infrastructure investments, land use, and human behavior there is a need for high quality and reliable data and tools for decision-making related to urban transportaiton systems. The lack of accurate and availbale estimates of the health impacts of such decisions has limited the effectiveness of efforts to encourage non-motorized transport in the US and in other countries. In the last decade new tools, applying comparative risk assessment methodologies, have made progress toward quantifying the multple health impacts of changes to the transportation system and behaviors. 

ITHIM is one such statistical model that integrates data on active transport,
physical activity, fine particulate matter and greenhouse gas
emissions to provide estimates for the proportional change in
mortality and morbidity for given baseline and alternate travel
scenarios. The model has been used to calculate the health impacts of
walking and bicycling short distances usually traveled by car or
driving low-emission automobiles \cite{woodcock2013,maizlish2013}.

ITHIM uses a comparative risk assessment framework for the active
transport component of the model.  We improve on the existing
implementation of the active transport component by including the
distribution of non-travel-related activity, improving numerical
precision when computing the population attributable fraction and
creating a simple user-interface for the custom-built \R{}
package \package{}.

Here we present this implementation using publically-available datasets with coverage of the entire United States, and investigate the sensitivity of the model outputs to variability in the input parameters. The results of this work suggest that statistical modeling and tools like the custom-built \R{} package \package{} can improve the application of this useful methodology and increase its utility to both practicioners and researchers. 
