Comparative risk assessment exercises to quantify the health potential of active transportation have been performed in the United States on a limited basis. Current methodologies, which rely on entensive data collection activities and user expertise, have estimated health benefits for select metropolitan areas under a few alternative scenarios. These analyses have largely been conducted using the Intergrated Transport Health Impacts Model (IHTIM), which was developed in the UK and has been adapted for use in the US, primarility California. Here we present an implementation of one portion of ITHIM's comparative risk assessment framework to model the health impacts of modified physical activity patterns from biking and walking. Using the newly-developed ITHIM package available for the R Software platform publically-available datasets with coverage for the entire US are used to quantify changes in diability adjusted life years. We examine the sensitivity of outcomes for the nation to input parameters; active travel time, non-travel physical activity time, specifically. The revised R package implemention of ITHIM's physical activity component provides a robust and stable means of examining the potential for changes in transportation behavior in the US, and the distribution of those benefits across age and gender categories.

%% While increased active transportation, primarily cycling and walking,
%% is often held up as a goal for improved public health --- via reduced
%% chronic disease, improved air quality, and reductions in greenhouse
%% gas emissions — obtaining estimates for the quantiative impacts of
%% such shifts remains infrequent, if not altogether absent from related
%% policy action. Using U.S. metropolitan counties from seven states we
%% perform a comparative risk analysis between each regions and national
%% averages as the baseline.  We employ a portion of the Integrated
%% Transport and Health Impacts Model (ITHIM) to investigate the reduced
%% chronic disease burden of ($x$, $y$, $z$ outcomes) from changing
%% levels of physical activity with population shifts in the amount of
%% biking and walking done for transport. The analysis is informed by
%% regional travel parameters from the National Household Transportation
%% Survey, non-travel activity metrics from the American Time Use Survey,
%% and baseline disease burden from the Centers for Disease Control and
%% Prevention's WONDER database and the Global Burden of Disease dataset.
%% Our findings show that the California and New York State have travel
%% patterns that, if mirrored by the nation, would save the U.S. upwards
%% of fifty thousand disability adjusted life years.

%\subsection{Summary:}
%% \section{Availability:}

%% %% \package{} may be explored using the user interface avaiable
%% %% at \webpage{}.
%% The source code and data files for \package{} are publicly available
%% on GitHub via a link on \webpage{}.

% \section{Supplementary information:}

% See \webpage{} for links to the source code, manual, tutorial and
% links for more information about the ITHIM model.

% \section{Contact:}
% \href{javargo@wisc.edu}{javargo@wisc.edu}

