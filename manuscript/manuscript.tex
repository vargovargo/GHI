\documentclass{bioinfo}
\usepackage{url}
\copyrightyear{2016}
\pubyear{2016}

\newcommand{\col}[2][red]{\textcolor{#1}{#2}}
\newcommand{\bi}{\begin{itemize}}
\newcommand{\ei}{\end{itemize}}
\newcommand{\package}{\emph{ITHIM}}
\newcommand{\ithim}{\emph{ITHIM}}
\newcommand{\R}{\emph{R}}
\newcommand{\sgy}[1]{{\itshape\col{#1}}}
\newcommand{\XX}{\sgy{XX}}
\newcommand{\webpage}{\url{http://www.ithim.ghi.wisc.edu}}

\begin{document}
\firstpage{1}

\title[\package{}]{\package{}: an \R{} package for implemetation of
  the ITHIM model}
\author[Younkin \textit{et~al}]{Samuel G. Younkin$^{1}$,
  Jason Vargo$^{1}$\footnote{to whom correspondence should be addressed},
  $\ldots$
  \ and Jonathan Patz$^{1}$}
\address{$^{1}$Global Health Institute\\
University of Wisconsin{\textendash}Madison, Madison, WI USA\\
}
\history{Received on XXXXX; revised on XXXXX; accepted on XXXXX}

\editor{Associate Editor: XXXXXXX}

\maketitle

\begin{abstract}

\section{Summary:}
\col[blue]{We divided the U.S. metropolitan regions into 10 regions defined by
the Department of Health and Human Services (HHS1-10).  We performed a
comparative risk analysis for each of the regions using national
averages as the baseline.  We investigated the active transport
component of the ITHIM model by estimating regional travel parameters
using the National Household Transportation Survey.  The West coast
and the Northeast of the U.S. have travel patterns that, if mirrored
by the nation, would save the U.S. thousands to hundreds of thousands
DALYs.}

\section{Availability:}
\col[blue]{\package{} may be explored using the user interface avaiable
at \webpage{}.  The source code and data files for \package{} are
publicly available on GitHub at \webpage{}.}

\section{Supplementary information:}

\col[blue]{See \webpage{} for links to the source code, manual, tutorial and
links for more information about the ITHIM model.}

\section{Contact:}
\col[blue]{\href{javargo@wisc.edu}{javargo@wisc.edu}}
\end{abstract}


\section{Introduction}
\col{ITHIM is a statistical model that integrates data on travel patterns,
physical activity, fine particulate matter, GHG emissions, and disease
and injuries. Based on population and travel scenarios. The model has
been used to calculate the health impacts of walking and bicycling
short distances usually traveled by car or driving low-emission
automobiles \cite{woodcock2013,maizlish2013}.}

\col{The ITHIM model uses comparative risk assessment through which it
formulates a change in the disease burden, resulting from the shift in
the exposure distribution from a baseline scenario to an alternative
scenario.}

\col{Lorem ipsum dolor sit amet, consectetur adipisicing elit, sed do
eiusmod tempor incididunt ut labore et dolore magna aliqua. Ut enim ad
minim veniam, quis nostrud exercitation ullamco laboris nisi ut
aliquip ex ea commodo consequat. Duis aute irure dolor in
reprehenderit in voluptate velit esse cillum dolore eu fugiat nulla
pariatur. Excepteur sint occaecat cupidatat non proident, sunt in
culpa qui officia deserunt mollit anim id est laborum. Lorem ipsum
dolor sit amet, consectetur adipisicing elit, sed do eiusmod tempor
incididunt ut labore et dolore magna aliqua.}

\col{Lorem ipsum dolor sit amet, consectetur adipisicing elit, sed do
eiusmod tempor incididunt ut labore et dolore magna aliqua. Ut enim ad
minim veniam, quis nostrud exercitation ullamco laboris nisi ut
aliquip ex ea commodo consequat. Duis aute irure dolor in
reprehenderit in voluptate velit esse cillum dolore eu fugiat nulla
pariatur. Excepteur sint occaecat cupidatat non proident, sunt in
culpa qui officia deserunt mollit anim id est laborum.}

\begin{methods}

\section{Methods}

\col[blue]{\begin{itemize}
  \item The coefficient of variation was fixed at ??? for all strata
    and models.  Earlier ITHIM models estimated this as a function of
    total active tranpsort time to model the decrease in variance of
    active transport time that occurs for increasing means, i.e.,
    fitter communities have less variance in active transport time.
  \item The mean non-travel activity for the referent group was
    modeled as a variable.  The relative means across age-sex strata
    were fixed, and estimated using ATUS.
  \item The national DALYs were stimated using the attributable
    fraction provided by the comparitive risk assesment.  The
    estimates for disease burned came from WHO and their GBD study.
  \item Not all of the regions had the same sample size.  In fact a
    few of the regions have relativel low sample size.  Maybe we
    should put confidence intervals on our estimates.
  \item The ITHIM model stratifies the population by sex and age using
    eight age classes.  Within each stratum active transport time is
    modeled as log-normal with mean $\hat{\mu}_{i,j}$ which is
    estimated from the National Household Transportation Survey.
  \item Quintiles, in this case the quantiles for percentages
    $(0.1,0.3,0.5,0.7,0.9)$, are used to discretize the integral
    defined in the Comparative Risk Assesment (CRA)
  \item 
\end{itemize}}

The ITHIM model uses the comparative risk assesment framework 

\subsection{Data Sources}

\col[red]{\begin{itemize}
\item National Household Transportation Survey (NHTS)
\item American time Use Survey
\item CDC Wonder
\end{itemize}}

\col[red]{Lorem ipsum dolor sit amet, consectetur adipisicing)) elit, sed do
eiusmod tempor incididunt ut labore et dolore magna aliqua. Ut enim ad
minim veniam, quis nostrud exercitation ullamco laboris nisi ut
aliquip ex ea commodo consequat. Duis aute irure dolor in
reprehenderit in voluptate velit esse cillum dolore eu fugiat nulla
pariatur. Excepteur sint occaecat cupidatat non proident, sunt in
culpa qui officia deserunt mollit anim id est laborum. Lorem ipsum
dolor sit amet, consectetur adipisicing elit, sed do eiusmod tempor
incididunt ut labore et dolore magna aliqua. Ut enim ad minim veniam,
quis nostrud exercitation ullamco laboris nisi ut aliquip ex ea
commodo consequat. Duis aute irure dolor in reprehenderit in voluptate
velit esse cillum dolore eu fugiat nulla pariatur. Excepteur sint
occaecat cupidatat non proident, sunt in culpa qui officia deserunt
mollit anim id est laborum. Lorem ipsum dolor sit amet, consectetur
adipisicing elit, sed do eiusmod tempor incididunt ut labore et dolore
magna aliqua. Ut enim ad minim veniam, quis nostrud exercitation
ullamco laboris nisi ut aliquip ex ea commodo consequat.}

\col[red]{Lorem ipsum dolor sit amet, consectetur adipisicing elit, sed do
eiusmod tempor incididunt ut labore et dolore magna aliqua. Ut enim ad
minim veniam, quis nostrud exercitation ullamco laboris nisi ut
aliquip ex ea commodo consequat. Duis aute irure dolor in
reprehenderit in voluptate velit esse cillum dolore eu fugiat nulla
pariatur. Excepteur sint occaecat cupidatat non proident, sunt in
culpa qui officia deserunt mollit anim id est laborum. Lorem ipsum
dolor sit amet, consectetur adipisicing elit, sed do eiusmod tempor
incididunt ut labore et dolore magna aliqua. Ut enim ad minim veniam,
quis nostrud exercitation ullamco laboris nisi ut aliquip ex ea
commodo consequat. Duis aute irure dolor in reprehenderit in voluptate
velit esse cillum dolore eu fugiat nulla pariatur. Excepteur sint
occaecat cupidatat non proident, sunt in culpa qui officia deserunt
mollit anim id est laborum. Lorem ipsum dolor sit amet, consectetur
adipisicing elit, sed do eiusmod tempor incididunt ut labore et dolore
magna aliqua. Ut enim ad minim veniam, quis nostrud exercitation
ullamco laboris nisi ut aliquip ex ea commodo consequat.}

\end{methods}

\section{U.S. Metro-Regional Analysis}
\col[blue]{Lorem ipsum dolor sit amet, consectetur adipisicing elit, sed do
eiusmod tempor incididunt ut labore et dolore magna aliqua. Ut enim ad
minim veniam, quis nostrud exercitation ullamco laboris nisi ut
aliquip ex ea commodo consequat. Duis aute irure dolor in
reprehenderit in voluptate velit esse cillum dolore eu fugiat nulla
pariatur. Excepteur sint occaecat cupidatat non proident, sunt in
culpa qui officia deserunt mollit anim id est laborum. Lorem ipsum
dolor sit amet, consectetur adipisicing elit, sed do eiusmod tempor
incididunt ut labore et dolore magna aliqua. Ut enim ad minim veniam,
quis nostrud exercitation ullamco laboris nisi ut aliquip ex ea
commodo consequat. Duis aute irure dolor in reprehenderit in voluptate
velit esse cillum dolore eu fugiat nulla pariatur. Excepteur sint
occaecat cupidatat non proident, sunt in culpa qui officia deserunt
mollit anim id est laborum. Lorem ipsum dolor sit amet, consectetur
adipisicing elit, sed do eiusmod tempor incididunt ut labore et dolore
magna aliqua. Ut enim ad minim veniam, quis nostrud exercitation
ullamco laboris nisi ut aliquip ex ea commodo consequat.}


\begin{figure}[t]
    \centerline{\includegraphics[width=0.4\textwidth]{./figures/hhsmap}}
    \centerline{\includegraphics[width=0.4\textwidth]{./figures/fig2}}
    \caption{Change in total national DALYs for each regional
      transportation estimates.}\label{dalyFigure}
\end{figure}

\col[blue]{Lorem ipsum dolor sit amet, consectetur adipisicing elit, sed do
eiusmod tempor incididunt ut labore et dolore magna aliqua. Ut enim ad
minim veniam, quis nostrud exercitation ullamco laboris nisi ut
aliquip ex ea commodo consequat. Duis aute irure dolor in
reprehenderit in voluptate velit esse cillum dolore eu fugiat nulla
pariatur. Excepteur sint occaecat cupidatat non proident, sunt in
culpa qui officia deserunt mollit anim id est laborum. Lorem ipsum
dolor sit amet, consectetur adipisicing elit, sed do eiusmod tempor
incididunt ut labore et dolore magna aliqua. Ut enim ad minim veniam,
quis nostrud exercitation ullamco laboris nisi ut aliquip ex ea
commodo consequat. Duis aute irure dolor in reprehenderit in voluptate
velit esse cillum dolore eu fugiat nulla pariatur. Excepteur sint
occaecat cupidatat non proident, sunt in culpa qui officia deserunt
mollit anim id est laborum. Lorem ipsum dolor sit amet, consectetur
adipisicing elit, sed do eiusmod tempor incididunt ut labore et dolore
magna aliqua. Ut enim ad minim veniam, quis nostrud exercitation
ullamco laboris nisi ut aliquip ex ea commodo consequat.}

\section*{Acknowledgment}
\paragraph{Funding\textcolon} SGY is supported by$\ldots$ , JV, JP, $\ldots$

%\bibliographystyle{natbib}
%\bibliographystyle{bmc_article}
\bibliographystyle{mypapers}
\bibliography{ITHIM}
\end{document}
